\documentclass[12pt]{report} % Increased the font size to 12pt
\usepackage{epigraph}
\usepackage{geometry}

% Optional: customize the style of epigraphs
\setlength{\epigraphwidth}{0.5\textwidth} % Adjust the width of the epigraph
\renewcommand{\epigraphflush}{flushright} % Align the epigraph to the right
\renewcommand{\epigraphrule}{0pt} % No horizontal rule
\usepackage[most]{tcolorbox}
\usepackage{amsmath, amssymb, amsthm}
\usepackage{bbm}
\usepackage{graphicx}
\usepackage{caption}
\usepackage[utf8]{inputenc}
\usepackage{hyperref} % Added for hyperlinks
\usepackage{listings} % Added for code listings
\usepackage{color}    % Added for color definitions
\usepackage[super]{nth}
\usepackage{fancyhdr}
\usepackage{tikz}
\usepackage{cite}
\usepackage{algpseudocode}
\usetikzlibrary{shapes.geometric, arrows, positioning}

\tikzstyle{startstop} = [rectangle, rounded corners, minimum width=3cm, minimum height=1cm, text centered, draw=black, fill=red!30]
\tikzstyle{io} = [trapezium, trapezium left angle=70, trapezium right angle=110, minimum width=3cm, minimum height=1cm, text centered, draw=black, fill=blue!30]
\tikzstyle{process} = [rectangle, minimum width=3cm, minimum height=1cm, text centered, draw=black, fill=orange!30]
\tikzstyle{decision} = [diamond, minimum width=3cm, minimum height=1cm, text centered, draw=black, fill=green!30]
\tikzstyle{arrow} = [thick,->,>=stealth]

% Define the graphics path
%\graphicspath{{./Plots/}}

% Define the header and footer for general pages
\pagestyle{fancy}
\fancyhf{} % Clear all header and footer fields
\fancyhead{} % Initially, the header is empty
\fancyfoot[C]{\thepage} % Page number at the center of the footer
\renewcommand{\headrulewidth}{0pt} % No header line on the first page of chapters
\renewcommand{\footrulewidth}{0pt} % No footer line

% Define the plain page style for chapter starting pages
\fancypagestyle{plain}{%
  \fancyhf{} % Clear all header and footer fields
  \fancyfoot[C]{\thepage} % Page number at the center of the footer
  \renewcommand{\headrulewidth}{0pt} % No header line
}

% Apply the 'fancy' style to subsequent pages in a chapter
\renewcommand{\chaptermark}[1]{%
  \markboth{\MakeUppercase{#1}}{}%
}

% Redefine the 'plain' style for the first page of chapters
\fancypagestyle{plain}{%
  \fancyhf{}%
  \fancyfoot[C]{\thepage}%
  \renewcommand{\headrulewidth}{0pt}%
}

% Header settings for normal pages (not the first page of a chapter)
\fancyhead[L]{\slshape \nouppercase{\leftmark}} % Chapter title in the header
\renewcommand{\headrulewidth}{0.4pt} % Header line width on normal pages

\setlength{\headheight}{14.49998pt}
\addtolength{\topmargin}{-2.49998pt}
% Define colors for code listings
\definecolor{codegreen}{rgb}{0,0.6,0}
\definecolor{codegray}{rgb}{0.5,0.5,0.5}
\definecolor{codepurple}{rgb}{0.58,0,0.82}
\definecolor{backcolour}{rgb}{0.95,0.95,0.92}

% Setup for code listings
\lstdefinestyle{mystyle}{
    backgroundcolor=\color{backcolour},
    commentstyle=\color{codegreen},
    keywordstyle=\color{magenta},
    numberstyle=\tiny\color{codegray},
    stringstyle=\color{codepurple},
    basicstyle=\ttfamily\footnotesize,
    breakatwhitespace=false,
    breaklines=true,
    captionpos=b,
    keepspaces=true,
    numbers=left,
    numbersep=5pt,
    showspaces=false,
    showstringspaces=false,
    showtabs=false,
    tabsize=2
}

\lstset{style=mystyle}

% Definition of the tcolorbox for definitions
\newtcolorbox{definitionbox}[1]{
  colback=red!5!white,
  colframe=red!75!black,
  colbacktitle=red!85!black,
  title=#1,
  fonttitle=\bfseries,
  enhanced,
}

% Definition of the tcolorbox for remarks
\newtcolorbox{remarkbox}{
  colback=blue!5!white,     % Light blue background
  colframe=blue!75!black,   % Darker blue frame
  colbacktitle=blue!85!black, % Even darker blue for the title background
  title=Remark:,            % Title text for remark box
  fonttitle=\bfseries,      % Bold title font
  enhanced,
}

% Definition of the tcolorbox for examples
\newtcolorbox{examplebox}{
  colback=green!5!white,
  colframe=green!75!black,
  colbacktitle=green!85!black,
  title=Example:,
  fonttitle=\bfseries,
  enhanced,
}

% Definitions and examples will be put in these environments
\newenvironment{definition}
    {\begin{definitionbox}}
    {\end{definitionbox}}

\newenvironment{example}
    {\begin{examplebox}}
    {\end{examplebox}}

\geometry{top=1.5in} % Adjust the value as needed
% ----------------------------------------------------------------------------------------


\title{Astronomy in the SKA Era: SKA-low Mini Project}
\author{CRSiD: tmb76}
\date{University of Cambridge}

\begin{document}

\maketitle

\tableofcontents

\chapter*{Gain Calibration of a SKA-low station}

\section{Introduction}

In this mini project, an algorithm for the retrieval of gain solutions for a single SKA-low station is implemented. One SKA1-low station comprises 256 antennae that cover a frequency range of 50-350 MHz. The gain retrieval algorithm is used in order to calibrate the stations, to account for known instrumental effects which occur in the analog chain: Low-Noise Amplifiers (LNA), cables, and other analog components. Because it can be summarised into a series of linear transformations of the input signal, the gain calibration can be done with a single complex-valued gain for each antenna.

\section{Calibration Problem}

In this short section, equations defining the problem of calibration are listed. First, the voltage that is input of the analog chain, for antenna $i$, frequency $f$, and feed port $p$, is given by:

\begin{equation}
    v_{i, p} = G_{i} \mathbf{F}_{i,p}(\theta, \phi) \cdot \mathbf{E}(\theta, \phi)
\end{equation}

where $\theta$ and $\phi$ are the zenith and azimuth angle, respectively. $\mathbf{E}$ is the incident electric field from the sky. $\mathbf{F}_{i,p}$ is the Embedded Element Pattern (EEP) of antenna $i$ for feed port $p$. And finally, $G_{i}$ is the complex gain for antenna $i$.

Then comes the visibilities which are the time cross-correlation of the voltage signals from two antennae, $i$ and $j$, and feed port $p$. There are the measured visibilities $R_{ij,p}$ which can be modeled as $R_{ij,p} = G_{i}G^{*}_{j}M_{ij,p}$ where $M_{ij,p}$ are model visibilities and they are given by:

\begin{equation}
    M_{ij,p} = \int\int (\mathbf{F}_{i,p}(\theta, \phi) \cdot \mathbf{F}_{j,p}^{*}(\theta, \phi))T_{b}(\theta, \phi)e^{-j\mathbf{k} \cdot (\mathbf{r_{i}}-\mathbf{r_{j}})} \sin\theta d\theta d\phi
\end{equation}

Where $\mathbf{F}_{j,p}^{*}(\theta, \phi)$ is the complex conjugate of the EEP of antenna $j$ for feed port $p$, $T_{b}(\theta, \phi)$ is the brightness temperature of the sky, and $\mathbf{r}_{i}$ is the position of antenna i and $\mathbf{k}$ is the wavevector with wavenumber k such that: $\mathbf{k} = k \sin \theta \cos \phi \mathbf{\hat{x}} + k \sin \theta \sin \phi \mathbf{\hat{y}} + k \cos \theta \mathbf{\hat{z}}$.

By combining visibilities for all feed ports, the measured visibilities can be written as $R_{ij} = R_{ij, X} + R_{ij, Y}$ and the model visibilities: $M_{ij} = M_{ij, X} + M_{ij, Y}$. Thus, equations (1) and (2) can be written in matrix form as:

\begin{equation}
    \mathbf{R} = \mathbf{G} \mathbf{M} \mathbf{G}^H
\end{equation}

The calibration problem is to find the gains $G_{i}$ that minimize the difference between the measured visibilities and the model visibilities. And this, taking equation (3), can be written as:

\begin{equation}
    \mathbf{\hat{G}} = \arg\min_{\mathbf{G_{i}}} ||\mathbf{R} - \mathbf{G} \mathbf{M} \mathbf{G}^H||_{F}
\end{equation}

Where $\mathbf{\hat{G}}$ is therefore an estimator of the gain that solves Eq (3), and $||\cdot||_{F}$ is the Frobenius norm of the matrix.

\chapter{Questions}

\section{Equation (4) and multiplying all gains by the same phase factor}

Equation (4) summarises the calibration problem. It states that the estimate for the gain solution should minimise the difference between the measured visibilities and the modeled visibilities, with respect to the gains $\mathbf{G_{i}}$. The Frobenius norm is used to measure this difference between the two matrices and can be described as the squareroot of the total squared error.

Now, from Eq (1) and (2), both the received and modelled visibilities are dependent on the sky being observed. Thus, in the context of calibration, the sky has to be known in order to solve for the gains.

All matrices in Eq. (4) are of size 256$\times$256. The gains matrix is diagonal, with diagonal elements being $G_{i}$, the gain for antenna $i$. Measured and modelled visibilities matrices are complex valued, and symmetric, since the cross-correlation of two antennae should be undirected.

By applying the Hermitian transpose to the gains matrix, it ensures the gains of each antenna is used.

\section{Power of Embedded Element Patterns (EEPs) and the Average Element Pattern (AEP)}

Here, we used the \texttt{compute\_EEP} function given in the starter code to plot the EEPs of the 256 antennae, as well the Average Element Pattern (AEP). Compute EEP description?.



\begin{figure}[h]
    \centering
    \includegraphics[width=0.7\textwidth]{../Plots/EEP_AEP_polX.png}
    \caption{The EEPs of the 256 antennae.}
    \label{fig:EEPs}
\end{figure}

Comments on Variability


\section{The StEFCal Algorithm}

Discussion of StEFCal algorithm . It solves system of eq 4 and finds per-antenna gain.




\end{document}
